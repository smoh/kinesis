\documentclass[modern,manuscript]{aastex61}

% to-do list
% ----------
% - Add items here

% style notes
% -----------
% - This file generates by Makefile; don't be typing ``pdflatex'' or some bullshit.
% - Line break between sentences to make the git diffs readable.
% - Use \, as a multiply operator.
% - Reserve () for function arguments; use [] or {} for outer shit.
% - Use \sectionname not Section, \figname not Figure, \documentname not Article or Paper or paper.


% packages
\definecolor{cbblue}{HTML}{3182bd}
\usepackage{microtype}  % ALWAYS!
\usepackage{amsmath,amssymb,natbib}
\hypersetup{backref,breaklinks,colorlinks,urlcolor=cbblue,linkcolor=cbblue,citecolor=black}

% graphicspath specifies where to look for figures. It is important to include
% the trailing / as this rewrites the includegraphics path to start with the
% graphics path specified above, if the file exists.
% Relative paths recommended.
\graphicspath{{./figures/}}


% define macros for text
\newcommand{\documentname}{Article}
\newcommand{\sectionname}{Section}
\newcommand{\figname}{Figure}
\newcommand{\eqname}{Equation}
\newcommand{\etal}{\textit{et al}.}
\newcommand{\project}[1]{\textsl{#1}}
\newcommand{\acronym}[1]{{\small{#1}}}
\newcommand{\gaia}{\project{Gaia}}
\newcommand{\rave}{\project{\acronym{RAVE}}}
\newcommand{\apogee}{\project{\acronym{APOGEE}}}
\newcommand{\tmass}{\project{\acronym{2MASS}}}
\newcommand{\dr}{\acronym{DR1}}
\newcommand{\tgas}{\acronym{TGAS}}


% define macros for math
\newcommand{\kms}{\ensuremath{\rm km~s^{-1}}}
\newcommand{\msun}{{\rm M}_\odot}
\newcommand{\pc}{{\rm pc}}
\newcommand{\given}{\,|\,}
\newcommand{\normal}{{\mathcal{N}}}
\newcommand{\dd}{\mathrm{d}}
\newcommand{\transp}[1]{{#1}^{\!\mathsf{T}}}
\newcommand{\inv}[1]{{#1}^{-1}}
\newcommand{\bs}[1]{\boldsymbol{#1}}
\newcommand{\vperp}{\bs{v}^\perp}
\newcommand{\propm}{\bs{\mu}}
\newcommand{\mat}[1]{\mathbf{#1}}
\renewcommand{\vec}[1]{\bs{#1}}
\newcommand{\data}{\mathrm{data}}
\newcommand{\snr}{[S/N]_\varpi}

\newcommand{\todo}[1]{{\color{blue}TODO:#1}}


\begin{document}\sloppy\sloppypar\raggedbottom\frenchspacing % trust me


\title{ Kinematic modelling of clusters with Gaia }

\author{
  Semyeong Oh
}

\begin{abstract}
  Abstract goes here
\end{abstract}

\section{Introduction}
\label{sec:introduction}




\begin{equation}
  \begin{split}
    \vec{u_i} = \vec{v_0} + \mat{T} \cdot (\vec{b_i}-\vec{b_0}) + \vec{\sigma}_i \\
    \vec{v_0} = \mathrm{constant},\,\, \vec{\sigma}_i \sim \normal(\vec{0},\,\sigma^2\mat{I})
  \end{split}
\end{equation}
The proper motion is
\begin{equation}
  \mu_i = \pi_i \vec{\hat\mu}_i^T \cdot \vec{u}_i
    = \pi_i \vec{\hat\mu}_i^T \cdot \left(\vec{v}_0 + \mat{T}\cdot(\vec{b}_i-\vec{b}_0)+\vec{\sigma}_i\right)
\end{equation}
The residual proper motion is
\begin{equation}
  \Delta{\mu} = \vec{\mu}_i/\pi_i - \vec{\hat\mu}_i^T \cdot \vec{v}_0 =
    \vec{\hat\mu}_i^T \cdot \left[ \mat{T}\cdot(\vec{b}_i-\vec{b}_0) + \vec{\sigma}_i\right]
\end{equation}
If $\mat{T}=\mat{0}$,
\begin{equation}
  \Delta\mu = \vec{\hat\mu}_i^T \vec{\sigma}_i
\end{equation}
For equatorial coordinates,
\begin{equation}
  \begin{split}
    \vec{\hat\mu}_{i,\,\alpha}^T = (-\sin\alpha_i,\,\cos\alpha_i,\,0) \\
    \vec{\hat\mu}_{i,\,\delta}^T = (-\sin\delta_i \cos\alpha_i,\,-\sin\delta_i \sin\alpha_i,\,\cos\delta_i) \\
  \end{split}
\end{equation}


There are variants of inference on the same key assumption that
a group of stars are comoving.

\noindent{\bf Moving cluster method}: If we know proper motions and radial velocities of stars
($\mu_i$ and $\vec{v}_0$), we can infer parallaxes $\pi_i$.

\noindent{\bf Astrometric radial velocity}:
If we have astrometric parameters ($\mu_i$ and $\pi_i$) measured for a group of
stars that you can assume to have an identical 3D velocity, because the
velocity vector projects slightly differently for different positions on the
celestial sphere, we can infer the missing (radial) velocity component ($\vec{v}_0$).


\section{Case in point: the Hyades cluster}

Given the \gaia\ data which improves the data quality,
are more complicated models where $\mat{T}\neq\mat{0}$ warranted?




% \bibliography{ref}

\end{document}
